\documentclass[11pt]{article}
\usepackage[margin=1in]{geometry}
\usepackage{amsmath, amssymb, amsthm}
\usepackage{mathtools}
\usepackage{lmodern}
\usepackage[T1]{fontenc}
\usepackage{hyperref}
\hypersetup{colorlinks=true, linkcolor=blue, urlcolor=blue}
\title{CIPH7 v1: An Invertible Byte-Wise Cipher Pipeline}
\author{Challenge Spec}
\date{}
\begin{document}
\maketitle

\section*{Overview}
We define an 8-step, fully invertible transformation over byte strings. Let a plaintext be a byte vector \(m = (m_0,\dots,m_{n-1})\) with each \(m_i \in \mathbb{Z}_{256} = \{0,\dots,255\}\). The cipher is keyed by a 32-bit seed \(s \in \{0,\dots,2^{32}-1\}\). All arithmetic is performed in the ring \(\mathbb{Z}_{256}\) unless otherwise stated.

The ciphertext is distributed as a single ASCII line
\[
\texttt{CIPH7:1:SEED=}0x\text{XXXXXXXX}\texttt{:LEN=}n'\texttt{:DATA=}\operatorname{b64}(c)
\]
where \(c\) is the final byte string after Step 7, and \(\operatorname{b64}\) denotes standard Base64 encoding.

\section*{PRNG}
We use the 32-bit xorshift PRNG initialized with seed \(s\):
\begin{align*}
X &\leftarrow s \\
X &\leftarrow X \oplus (X \ll 13) \\
X &\leftarrow X \oplus (X \gg 17) \\
X &\leftarrow X \oplus (X \ll 5)
\end{align*}
The state \(X\) evolves per call; output bytes are \(X \bmod 256\). For unbiased indices in Fisher--Yates we use rejection sampling over 32-bit outputs.

\section*{Checksum Prefix}
Compute two checksums over the plaintext \(m\):
\[
 c_0 = \sum_{i=0}^{n-1} m_i \pmod{256}, \qquad c_1 = \sum_{i=0}^{n-1} i\,m_i \pmod{256}.
\]
Form the working vector by prefixing them: \(w^{(0)} = (c_0, c_1, m_0,\dots,m_{n-1})\).

\section*{Step 1: Affine Byte Map}
Let
\[
 a = 1 + 2\big((s \bmod 127) + 1\big) \bmod 254, \qquad b_0 = \big\lfloor s/2^8 \big\rfloor \bmod 256.
\]
Note that \(a\) is odd, hence invertible modulo \(2^8\). For each position \(i\) apply
\[
 w^{(1)}_i = a\,w^{(0)}_i + b_0 + i \pmod{256}.
\]

\section*{Step 2: Global Permutation}
Generate a permutation \(\pi\) of \(\{0,\dots,|w^{(1)}|-1\}\) using Fisher--Yates where the indices are driven by the PRNG. Apply it as
\[
 w^{(2)}_i = w^{(1)}_{\pi(i)}.
\]
\(\pi\) is bijective; its inverse \(\pi^{-1}\) is used during decryption.

\section*{Step 3: Nonlinear S-box}
Derive constants
\[
 r_1 = \big\lfloor s/2^{24} \big\rfloor \bmod 256, \qquad r_2 = 1 + \big( \big\lfloor s/2^{16} \big\rfloor \bmod 8 \big).
\]
Define the S-box as XOR followed by 8-bit rotation:
\[
 S(x) = \operatorname{ROTL}_8\big(x \oplus r_1,\ r_2\big).
\]
Apply component-wise: \(w^{(3)}_i = S(w^{(2)}_i)\).

\section*{Step 4: Keystream XOR}
Generate a keystream \(k_i\) from successive PRNG bytes and apply
\[
 w^{(4)}_i = w^{(3)}_i \oplus k_i.
\]

\section*{Step 5: Running Sum with IV}
Let the IV be \(\text{iv} = (s \oplus \text{0xA5A5A5A5}) \bmod 256\). Define
\[
 w^{(5)}_0 = w^{(4)}_0 + \text{iv} \pmod{256}, \qquad w^{(5)}_i = w^{(4)}_i + w^{(5)}_{i-1} \pmod{256}\ \ (i\ge 1).
\]

\section*{Step 6: Chunk Reversal}
Let \(L = 3 + (s \bmod 5)\). Partition \(w^{(5)}\) into consecutive chunks of length \(L\) (the last chunk may be shorter) and reverse the order of elements within each chunk to obtain \(w^{(6)}\).

\section*{Step 7: Output}
Let \(c = w^{(6)}\). Armor using Base64 and attach the header with version, seed, and length.

\section*{Decryption}
All steps are invertible; apply in reverse order:
\begin{enumerate}
  \item Undo chunk reversal with the same \(L\).
  \item Undo running sum: \(x_0 = w^{(5)}_0 - \text{iv}\), \(x_i = w^{(5)}_i - w^{(5)}_{i-1}\) for \(i\ge 1\).
  \item XOR the same keystream.
  \item Apply inverse S-box: \(S^{-1}(y) = \operatorname{ROTR}_8(y, r_2) \oplus r_1\).
  \item Apply \(\pi^{-1}\).
  \item Apply inverse affine map. Let \(a^{-1} \equiv a^{-1} \pmod{256}\). For each index \(i\):
  \[
   w^{(0)}_i = a^{-1}\big(w^{(1)}_i - b_0 - i\big) \pmod{256}.
  \]
  \item Remove the two-byte prefix and verify \(c_0, c_1\) against the recovered message.
\end{enumerate}

\section*{Notes}
\begin{itemize}
  \item The affine multiplier \(a\) is guaranteed odd, hence invertible in \(\mathbb{Z}_{256}\).
  \item Fisher--Yates with rejection-sampled indices yields a uniform permutation conditional on the PRNG outputs.
  \item All state used during encryption (seed, length) is disclosed in the header, so the construction is a puzzle rather than a secure cryptosystem.
\end{itemize}

\end{document}
